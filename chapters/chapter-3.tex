\newpage
\chapter{METODE PENELITIAN} \label{Bab III}

\section{Alur Penelitian} \label{III.Alur}
Pada penelitian ini, alur dirancang untuk memastikan setiap tahapan pemrosesan dilakukan secara sistematis dan efisien. Alur penelitian ini mencerminkan langkah-langkah utama yang dilakukan pada penelitian ini, dapat dilihat pada Gambar \ref{fig:3.alur}. \par

\begin{figure}[H] % Kalau menggunakan H, posisi gambar akan tepat dibawah teks
    \centering
    \includegraphics[width=0.6\textwidth]{figure/samplephoto.jpg}
    \caption{Alur Penelitian}
    \label{fig:3.alur}
\end{figure}

\section{Penjabaran Langkah Penelitian} \label{III.Jabar Alur}
Untuk memperjelas setiap langkah-langkah yang telah didefinisikan pada Gambar \ref{fig:3.alur}, berikut ini akan dijelaskan secara rinci tahapan-tahapan yang dilakukan dalam penelitian ini.

\subsection{Identifikasi Masalah} \label{III.Identifikasi_masalah}
\lipsum[1] % Menampilkan paragraf 1 sampai 2 dari lorem ipsum
\begin{equationcaptioned}[eq:3.mae]{
		MAE = \frac{1}{n} \sum_{i=1}^{n} \left| y_i - \hat{y}_i \right|
	}{
		Mean Absolute Error (MAE)
	}
\end{equationcaptioned}

\section{Alat dan Bahan Tugas Akhir} \label{III.Alat dan Bahan}
Dalam menjalani penelitian, beberapa alat dan bahan digunakan untuk memastikan penelitian berjalan dengan baik.\par

\subsection{Alat} \label{III.Alat}
Dalam membuat pengukuran frekuensi denyut nadi non-kontak dalam penelitian, berikut adalah alat-alat yang digunakan: \par
\begin{enumerate}[noitemsep]
	\item \textit{Visual Studio Code} sebagai \textit{tools} untuk \textit{text editor}.
	\item Python versi 3.12.5
        \item OpenCV versi 4.10.0.84
	\item NumPy versi 2.1.1
	\item Mediapipe versi 0.10.14
	\item Scipy versi 1.12.0
        \item Matplotlib versi 3.8.3
        \item Flask versi 3.1.0
\end{enumerate}

\subsection{Bahan} \label{III.Bahan}
Dataset yang digunakan pada penelitian ini merupakan dataset 

\section{Ilustrasi Perhitungan Metode} \label{III.Ilustrasi}
Dalam penelitian ini, hasil perhitungan dari program akan melalui serangkaian pengujian untuk mengevaluasi tingkat keakuratan model yang digunakan. Data dummy tersebut dapat dilihat pada Tabel \ref{table:3.dummy}. \par

\begin{longtable}{|c|c|c|c|c|c|c|c|c|}
	\caption{Data \textit{dummy} Pengujian}
	\label{table:3.dummy}\\
	\hline
	\multirow{2}{*}{\textbf{Subjek}} & \multicolumn{7}{|c|}{\textbf{Hasil Prediksi (BPM)}} & \multirow{2}{*}{\textbf{GT}} \\ \cline{2-8}
    & \textbf{F} & \textbf{NA} & \textbf{NO} & \textbf{RC} & \textbf{LC} & \textbf{M} & \textbf{C} & \\ 
        \hline
	   \endfirsthead
       \hline
       \multirow{2}{*}{\textbf{Subjek}} & \multicolumn{7}{|c|}{\textbf{Hasil Prediksi (BPM)}} & \multirow{2}{*}{\textbf{GT}} \\ \cline{2-8}
    & \textbf{F} & \textbf{NA} & \textbf{NO} & \textbf{RC} & \textbf{LC} & \textbf{M} & \textbf{C} & \\ 
    \hline
	\endhead
	\hline
	\endfoot
	\hline
	\endlastfoot
	1 & 68 & 69 & 68 & 70 & 68 & 71 & 69 & 68 \\ 
	\hline
	2 & 69 & 69 & 68 & 70 & 68 & 71 & 69 & 69 \\
	\hline
	3 & 70 & 70 & 69 & 71 & 68 & 73 & 69 & 70\\
	\hline
	4 & 71 & 70 & 70 & 72 & 69 & 73 & 70 & 71 \\
	\hline
	5 & 72 & 72 & 70 & 72 & 70 & 74 & 70 & 72 \\
	\hline
        6 & 73 & 72 & 71 & 74 & 71 & 76 & 71 & 73 \\ 
	\hline
	7 & 74 & 73 & 72 & 74 & 72 & 77 & 71 & 74 \\
	\hline
	8 & 75 & 74 & 72 & 74 & 73 & 77 & 73 & 75\\
	\hline
	9 & 76 & 75 & 73 & 75 & 74 & 78 & 75 & 76 \\
	\hline
	10 & 77 & 76 & 74 & 78 & 75 & 78 & 76 & 77
\end{longtable}