\newpage
\pagestyle{fancy}
\fancyhf{}
\fancyhead[R]{\thepage}
\chapter{PENDAHULUAN} \label{Bab I}

\section{Latar Belakang} \label{I.Latar Belakang}
Latar Belakang berisi dasar pemikiran, kebutuhan atau alasan yang menjadi ide dari topik tugas akhir. Tujuan utamanya adalah untuk memberikan informasi secukupnya kepada pembaca agar memahami topik yang akan dibahas. Terdapat dua hal yang wajib dikemukakan: \par

\begin{enumerate}[noitemsep]
	\item Deskripsi yang luas dan longgar yang berkaitan dengan bidang/masalah di masyarakat, industry dan atau bidang-bidang lainnya. Deskripsi ini mewakili bidang/masalah secara umum yang berkaitan dengan Teknik Informatika, bekerja dan akan terlibat di dalamnya. Sangat disarankan di sini, sebisa mungkin tidak ada Batasan tentang pilihan teknologi yang akan digunakan. Contoh: bidang transportasi, bidang telekomunikasi, bidang Pendidikan, bidang manufaktur, bidang renewable energi, pariwisata, militer, transportasi, kesehatan, pertanian, pengelolaan infrastruktur dan sebagainya.
	\item Deskripsi lebih khusus dan mendetail yang didapatkan dari poin 1 di atas. Dari deskripsi umum di atas, selanjutkan fokuskan pada fenomena masalah yang akan diangkat. Pendetailan harus mampu membawa masalah kepada masalah yang mennjukkan peran Anda dalam penelitian 
\end{enumerate}

\section{Rumusan Masalah} \label{I.Rumusan Masalah}
Merumuskan masalah secara konkrit, bentuk pertanyaan fakta / kebenaran yang masih dipertanyakan \par

Dari pendahuluan di atas, mahasiswa diharapkan dapat memformulasikan masalah engineering yang solid. Masalah yang kemudian akan diformulasi mahasiswa harus terdefinisi dengan baik (harus jelas, tidak ambigu/ada makna ganda, tanpa menggunakan jargon), masalah harus real (benar-benar ada masalah terebut) sehingga nantinya akan ada solusi yang konkrit. Perlu dipertimbangkan juga masalah tersebut harus bisa dipecahkan dalam waktu maksimal 1 semester oleh mahasiswa dengan alokasi waktu per minggu tidak lebih dari 20 jam per minggu. \par

Lebih jelasnya masalah yang diharapkan adalah seperti dalam 3 poin di bawah ini. Jika tidak mengandung semua unsur dibawah maka tugas akhir ini tidak memenuhi syarat sebagai tugas akhir. \par

\begin{enumerate}[noitemsep]
	\item Harus ada proses perancangan yang utuh dari penentuan masalah real yang perlu dipecahkan, 
	\item Harus menjelaskan spesifikasi yang akan dibuat
	\item Harus ada implementasi dalam bentuk salah satu di bawah ini:
	\begin{enumerate}
		\item Hardware/perangkat keras
		\item Software/perangkat lunak
		\item Proses/simulasi yang dibuat sendiri (Matlab, C/C++, Python, dan lain-lain) bukan melalui software yang murni dan sudah paten dan tinggal memasukkan data (ETAP, EDSA, SPSS, dan lain-lain)
	\end{enumerate}
\end{enumerate}

Hasil rancangan dalam bentuk hardware/software/simulasi tersebut harus diuji dan diverifikasi apakah bekerja dengan baik atau belum Jika belum bekerja baik, mahasiswa harus bisa menjelaskan alasannya dan perbaikannya ke depan (walau pun saat tugas akhir ini selesai, alat/software/simulasi belum bisa bekerja).\par

Selain itu, rumusan sangat disarankan untuk melibatkan pengalaman multidisiplin. Misalnya melibatkan unsur-unsur seperti seni, ekonomi, mekanik, politik, proses kimia, etika, kesehatan, dan sebagainya. Contoh-contoh rumusan masalah yang \textbf{tidak disarankan}: \par

\begin{enumerate}[noitemsep]
	\item \textbf{Masalah tidak real dan tidak terlalu hipotetis}. Misalnya, topik riset atau topik untuk lomba (contoh: mencari metode paling cepat untuk menentukan posisi kebakaran di dalam hutan).
	\item Rumusan untuk membuat alat/produk yang \textbf{tidak dapat diimplemetasikan dan diukur/diuji dalam waktu maksimal 2 semester}. Misalnya membuat roket dengan daya jangkau 500 km.
	\item \textbf{Solusi terlalu kompleks} sehingga dalam satu tahun hanya dapat menghasilkan bagian kecil dari solusi yang diharapkan Rumusan masalah berisi ringkasan fenomena dan masalah.
\end{enumerate}

Rumusan masalah akan dijawab di \nameref{V.Kesimpulan}. Rumusan Masalah disarankan untuk ditulis per poin.

\section{Tujuan Penelitian} \label{I.Tujuan}
Tujuan diisikan tujuan dari penelitian yang dilakukan, berdasarkan sub-bab \nameref{I.Latar Belakang} dan \nameref{I.Rumusan Masalah} dilengkapi dengan spesifikasinya. Tuliskan Tujuan sesuai dengan poin Rumusan Masalah. \par

\section{Batasan Masalah} \label{I.Batasan}
Batasan yang dimaksud disini ialah batasan dari penelitian tugas akhir yang dilakukan. Batasan masalah ditujukan agar tugas akhir yang dilakukan tidak terlalu luas, dan menjadi lebih realistis untuk diselesaikan. \par

\section{Manfaat Penelitian} \label{I.Manfaat}
Manfaat tugas akhir yang dilakukan didefinisikan sebagai manfaat yang diperoleh ketika tugas akhir telah selesai dilakukan. Manfaat dapat berupa manfaat untuk: \textbf{mahasiswa, program studi teknik informatika, ITERA, masyarakat, dunia akademisi, dan dunia penelitian}. \par


\section{Sistematika Penulisan} \label{I.Sistematika}
Sistematika penulisan berisi pembahasan apa yang akan ditulis disetiap Bab. Sistematika pada umumnya berupa paragraf yang setiap paragraf mencerminkan bahasan setiap Bab. \par

\subsection*{Bab I}
Bab ini berisikan penjelasan latar belakang dari topik penelitian yang berlangsung, rumusan masalah dari masalah yang dihadapi pada penjelasan di latar belakang, tujuan dari penelitian, batasan dari penelitian, manfaat dari hasil penelitian, dan sistematika penulisan tugas akhir. \par

\subsection*{Bab II}
Bab ini membahas mengenai tinjauan pustaka dari penelitan terdahulu dan dasar teori yang berkaitan dengan penelitian ini.

\subsection*{Bab III}
Bab ini berisikan penjelasan alur kerja sistem, alat dan data yang digunakan, metode yang digunakan, dan rancangan pengujian.

\subsection*{Bab IV}
Bab ini membahas hasil implementasi dan pengujian dari penelitian yang dilakukan, serta analisis dan evaluasi yang dapat dipetik dari hasil.

\subsection*{Bab V}
Bab ini membahas kesimpulan dari hasil penelitian dan juga saran untuk penelitian selanjutnya.